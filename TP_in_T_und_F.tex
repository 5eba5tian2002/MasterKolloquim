\documentclass[11pt]{beamer}

% ============================
%   BEAMER THEME & COLORS
% ============================
\usetheme{Madrid}
\usecolortheme{default}

% ============================
%   SPRACHE & ENCODING
% ============================
\usepackage[ngerman]{babel}
\usepackage[utf8]{inputenc}
\usepackage[T1]{fontenc}

% ============================
%   MATHE & SYMBOLE
% ============================
\usepackage{amsmath}
\usepackage{amssymb}
\usepackage{siunitx}
\usepackage{gensymb}

% ============================
%   GRAFIKEN & TIKZ
% ============================
\usepackage{graphicx}
\usepackage{tikz}
\usetikzlibrary{arrows.meta, positioning, shapes.geometric, mindmap, trees}

% ============================
%   TABELLEN
% ============================
\usepackage{tabularx}
\usepackage{array}
\usepackage{multirow}

% ============================
%   HYPERREF
% ============================
\usepackage{hyperref}
\hypersetup{
    colorlinks=true,
    linkcolor=blue,
    urlcolor=cyan
}

% ============================
%   EIGENE KOMMANDOS
% ============================
\newcommand*{\quelle}[1]{\par\raggedleft\footnotesize Quelle:~#1}

% ============================
%   TITELSEITE
% ============================
\title{Beschreibung von Tiefpassfiltern\\im Zeit- und Frequenzbereich}
\subtitle{Masterkolloquium Elektrotechnik}
\author{Sebastian Pasinski}
\institute{TH Aschaffenburg}
\date{\today}

% ============================
%   DOKUMENT
% ============================
\begin{document}

% Titelseite
\begin{frame}
    \titlepage
\end{frame}

\begin{frame}{Gliederung der Präsentation}
    \tableofcontents
\end{frame}


\section{Definition Tiefpass}

\begin{frame}{Definition Tiefpass}

	\begin{figure}
        \centering
        \includegraphics[width=0.8\textwidth]{images/FilterungTP.png}
        \caption{Filterung eines Sinussignals mit einem RC-Tiefpass}
    \end{figure}
    
\end{frame}






\section{Tiefpass im Zeitbereich}



\begin{frame}{Tiefpass im Zeitbereich}

 Systembeschreibung mit DGL
% Noch Sichereren Dialog erstellen!!!!

\begin{minipage}{0.58\textwidth}
    Maschengleichung
    \[
        -u_{\mathrm{e}}(t) + u_{\mathrm{R}}(t) + u_{\mathrm{C}}(t) = 0
    \]

    % Stromgleichung

% Ohmsches Gesetz

% Kondensatorgesetz

%\subsection*{Einsetzen in die Maschengleichung}

%\subsection*{Differentialgleichung des RC-Tiefpasses}

    Differentialgleichung
    \[
        RC\, \frac{\mathrm{d}u_{\mathrm{C}}(t)}{\mathrm{d}t} + u_{\mathrm{C}}(t) = u_{\mathrm{e}}(t)
    \]

\cite{SigSys:2022}

    %Integration beider Seiten:


%Umgestellt nach der Ausgangsspannung:
\end{minipage}
\hfill
\begin{minipage}{0.38\textwidth}
    \centering
    \includegraphics[width=\textwidth]{images/RC-Schaltung.png}

    {\footnotesize RC‑Tiefpass-Schaltung \par}
    {\footnotesize Quelle: \cite{wikipediaRCGlied:2026} \par}
\end{minipage}

Berechnungen von DGL oft komplex und aufwendig

\end{frame}

\begin{frame}{Sprungantwort}

        \centering
        \includegraphics[width=\textwidth]{images/SprungantwortRC.png}
        {\footnotesize Sprungsantwort RC‑Tiefpass \par}
    

\cite{TietzeSchenk:2019}

\end{frame}

\begin{frame}{Impulsantwort}

    \centering
    \includegraphics[width=\textwidth]{images/ImpulsantwortRC.png}
    {\footnotesize Impulsantwort RC‑Tiefpass \par}

\cite{SigSys:2022}

\end{frame}


\begin{frame}{Darstellung von Signalen als Impulsfolge}
    
    \begin{figure}
        \centering
        \includegraphics[width=0.8\textwidth]{images/SinusDirac.png}
        \caption{Sinus als Dirac-Impulsfolge}
    \end{figure}
\cite{Faltung:2026}
    
\end{frame}

\begin{frame}{Übertragungsfunktion und Impulsantwort}

    Impulsantwort ist die Inverse Laplace-Transformation der Übertragungsfunktion:
\[
H(s) = \mathcal{L}\{h(t)\}
\]

\[
h(t) = \mathcal{L}^{-1}\{H(s)\}
\]

\cite{SigSys:2022}

\end{frame}


\begin{frame}{Faltung als Beschreibung im Zeitbereich}
    Berechnung Ausgangssignal mit Eingangssignal und Impulsantwort: 

\[
y(t) = (x * h)(t)
\]

\[
y(t) = \int_{-\infty}^{\infty} x(\tau)\, h(t - \tau)\, d\tau
\]




\cite{Faltung:2026}


\begin{figure}
        \centering
        \includegraphics[width=0.8\textwidth]{images/Faltung.png}
        
        {\footnotesize Graphische Faltung \par}
    {\footnotesize Quelle: \cite{GraphischeFaltung:2024} \par}
    \end{figure}

%Signal spiegeln und verschiebern
% beide signal multiplizieren und integrieren über alle Zeitpunkte


\end{frame}














\section{Tiefpass im Frequenzbereich}
\begin{frame}{Beschreibung im Frequenzbereich: Übertragungsfunktion}


\begin{columns}[T]  % T = top aligned
    % Linke Spalte
    \begin{column}{0.55\textwidth}
        \[
        \frac{U_\text{c}}{U_\text{e}} = \frac{Z_C}{Z_R + Z_C}
        \]

        \[
        \frac{U_\text{c}}{U_\text{e}} = \frac{\frac{1}{j\omega C}}{R + \frac{1}{j\omega C}}
        \]

        \[
        \frac{U_\text{c}}{U_\text{e}} = \frac{1}{1 + j\omega RC}
        \]
        \center

        Spannungsverstärkung abhängig von der Frequenz
       \cite{TietzeSchenk:2019}
    \end{column}

    % Rechte Spalte
    \begin{column}{0.55\textwidth}
        \begin{figure}
            \centering
            \includegraphics[width=0.8\textwidth]{images/ÜFunc_Schaltung.png}
            {\footnotesize \cite{ÜfuncRC:2026} \par}
            \caption{RC-Schaltung}
        \end{figure}
    \end{column}
\end{columns}



\end{frame}

\begin{frame}{Beschreibung Systemverhalten mit Übertragungsfunktion}

Übertragungsfunktion:

\begin{itemize}

\item Spannungsverstärkung abhängig von der Frequenz
\item im Frequenzbereich kann Systemantwort bestimmen
\end{itemize}
 
\[
Y(\omega) = H(\omega)\, X(\omega)
\]

\cite{SigSys:2022}




\end{frame}



\begin{frame}{Grenzfrequenz}

Bei der Grenzfrequenz fällt die Ausgangsspannung auf $1/\sqrt{2}$ des Eingangswerts.
Kann aus Übertragungsfunktion berechnet werden:

\begin{enumerate}
    \item Betrag bilden:
    \[
    |H(j\omega)| = \frac{1}{\sqrt{1 + (\omega RC)^2}}
    \]

    \item $-3\,\mathrm{dB}$-Bedingung anwenden:
    \[
    |H(j\omega_c)| = \frac{1}{\sqrt{2}}
    \]

    \item Gleichsetzen und lösen:
    \[
    \frac{1}{\sqrt{1 + (\omega_c RC)^2}} = \frac{1}{\sqrt{2}}
    \quad\Rightarrow\quad
    \omega_c = \frac{1}{RC}
    \]
\end{enumerate}

\cite{GrundlagenEtechnik:2006}

\end{frame}

\begin{frame}{Amplitudengang eines RC-Tiefpasses}

\begin{columns}[T]  % T = top aligned
    % Linke Spalte
    \begin{column}{0.30\textwidth}

\center
      Dämpfung von Ein-zu Ausgang 
            
\[
    |H(\omega)| = \frac{1}{\sqrt{1 + (\omega RC)^2}}
\]

 bei fg : Amplitude fällt auf -3 dB


    \end{column}

    % Rechte Spalte
    \begin{column}{0.55\textwidth}
       
            \begin{figure}
            \centering
            \includegraphics[width=\textwidth]{images/AmplitudenGang.png}
   
            \caption{Amp.gang RC-TP}
        \end{figure}
    
    \end{column}
\end{columns}
\cite{TietzeSchenk:2019}

\end{frame}


\begin{frame}{Phasengang eines RC-Tiefpasses}

\begin{columns}[T]  % T = top aligned
    % Linke Spalte
    \begin{column}{0.30\textwidth}

\center
      Phasenverschiebung von Ein-zu Ausgang
            
\[
            \varphi(\omega) = -\arctan(\omega RC)
        \]

 bei fg : Verschiebung von -45°

% Niederige Frequenzen kaum verschiebung
% hohe Frequenzen verschiebung nähert sich 90°
% Phase hinkt hinterher, Kapazität Spannung zu spät




    \end{column}

    % Rechte Spalte
    \begin{column}{0.55\textwidth}
       
            \begin{figure}
            \centering
            \includegraphics[width=\textwidth]{images/PhasengangRC.png}
   
            \caption{Ph.gang RC-TP}
        \end{figure}
    
    \end{column}
\end{columns}

\cite{TietzeSchenk:2019}

\end{frame}



\begin{frame}{Bodediagramm}
Darstellung beider Verhalten auf einen Blick

    \begin{figure}
        \centering
        \includegraphics[width=0.8\textwidth]{images/BodediagrammRC_markiert.png}
        \caption{Bodediagramm eines RC-Tiefpasses}
    \end{figure}
\end{frame}

\section{Fazit}
%\input{sections/fazit}

\end{document}