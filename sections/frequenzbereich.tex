\begin{frame}{Beschreibung im Frequenzbereich: Übertragungsfunktion}


\begin{columns}[T]  % T = top aligned
    % Linke Spalte
    \begin{column}{0.55\textwidth}
        \[
        \frac{U_\text{c}}{U_\text{e}} = \frac{Z_C}{Z_R + Z_C}
        \]

        \[
        \frac{U_\text{c}}{U_\text{e}} = \frac{\frac{1}{j\omega C}}{R + \frac{1}{j\omega C}}
        \]

        \[
        \frac{U_\text{c}}{U_\text{e}} = \frac{1}{1 + j\omega RC}
        \]
        \center

        Spannungsverstärkung abhängig von der Frequenz
    \end{column}

    % Rechte Spalte
    \begin{column}{0.55\textwidth}
        \begin{figure}
            \centering
            \includegraphics[width=0.8\textwidth]{images/ÜFunc_Schaltung.png}
            {\footnotesize \cite{ÜfuncRC:2026} \par}
            \caption{RC-Schaltung}
        \end{figure}
    \end{column}
\end{columns}

\end{frame}

\begin{frame}{Grenzfrequenz}

\begin{itemize}
    \item Bei der Grenzfrequenz fällt die Ausgangsspannung auf $1/\sqrt{2}$ des Eingangswerts.
    \item Dadurch halbiert sich die Leistung ($P \propto U^2$).
    \item Dieser Punkt entspricht dem \(-3\,\mathrm{dB}\)-Pegel.
\end{itemize}

\end{frame}

\begin{frame}{Grenzfrequenz aus der Übertragungsfunktion (RC-Tiefpass)}

\begin{enumerate}
    \item Betrag bilden:
    \[
    |H(j\omega)| = \frac{1}{\sqrt{1 + (\omega RC)^2}}
    \]

    \item $-3\,\mathrm{dB}$-Bedingung anwenden:
    \[
    |H(j\omega_c)| = \frac{1}{\sqrt{2}}
    \]

    \item Gleichsetzen und lösen:
    \[
    \frac{1}{\sqrt{1 + (\omega_c RC)^2}} = \frac{1}{\sqrt{2}}
    \quad\Rightarrow\quad
    \omega_c = \frac{1}{RC}
    \]
\end{enumerate}

\end{frame}

\begin{frame}{Amplitudengang eines RC-Tiefpasses}

\begin{itemize}
    \item Konstante Verstärkung im Tieffrequenzbereich ($|H| \approx 1$).
    \item Abfall ab der Grenzfrequenz $f_c = \frac{1}{2\pi RC}$.
    \item Bei $f_c$: Amplitude fällt auf $1/\sqrt{2}$ ($-3\,\mathrm{dB}$).
\end{itemize}

\end{frame}

\begin{frame}{Phasengang eines RC-Tiefpasses}

\begin{itemize}
    \item Phase beginnt bei $0^\circ$ für niedrige Frequenzen.
    \item Bei der Grenzfrequenz: Phase = $-45^\circ$.
    \item Für hohe Frequenzen nähert sich die Phase $-90^\circ$.
\end{itemize}

\end{frame}



\begin{frame}{Tiefpass im Frequenzbereich}
    \begin{itemize}
        \item Beschreibung über die Übertragungsfunktion:
        \[
            H(\omega) = \frac{1}{1 + j\omega RC}
        \]
        \item Grenzfrequenz:
        \[
            \omega_c = \frac{1}{RC}
        \]
        \item Amplitudengang:
        \[
            |H(\omega)| = \frac{1}{\sqrt{1 + (\omega RC)^2}}
        \]
        \item Phasengang:
        \[
            \varphi(\omega) = -\arctan(\omega RC)
        \]
        \item Tiefpass dämpft hohe Frequenzen und lässt niedrige passieren
    \end{itemize}
     \begin{itemize}
        \item Übertragungsfunktion
        \item Amplitudengang
        \item Phasengang
        \item Bodediagramm
    \end{itemize}
\end{frame}


\begin{frame}{Bodediagramm}

    \begin{figure}
        \centering
        \includegraphics[width=0.8\textwidth]{images/BodediagrammRC_markiert.png}
        \caption{Bodediagramm eines RC-Tiefpasses}
    \end{figure}
\end{frame}