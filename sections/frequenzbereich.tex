\begin{frame}{Beschreibung im Frequenzbereich: Übertragungsfunktion}


\begin{columns}[T]  % T = top aligned
    % Linke Spalte
    \begin{column}{0.55\textwidth}
        \[
        \frac{U_\text{c}}{U_\text{e}} = \frac{Z_C}{Z_R + Z_C}
        \]

        \[
        \frac{U_\text{c}}{U_\text{e}} = \frac{\frac{1}{j\omega C}}{R + \frac{1}{j\omega C}}
        \]

        \[
        \frac{U_\text{c}}{U_\text{e}} = \frac{1}{1 + j\omega RC}
        \]
        \center

        Spannungsverstärkung abhängig von der Frequenz
       \cite{TietzeSchenk:2019}
    \end{column}

    % Rechte Spalte
    \begin{column}{0.55\textwidth}
        \begin{figure}
            \centering
            \includegraphics[width=0.8\textwidth]{images/ÜFunc_Schaltung.png}
            {\footnotesize \cite{ÜfuncRC:2026} \par}
            \caption{RC-Schaltung}
        \end{figure}
    \end{column}
\end{columns}



\end{frame}

\begin{frame}{Beschreibung Systemverhalten mit Übertragungsfunktion}

Übertragungsfunktion:

\begin{itemize}

\item Spannungsverstärkung abhängig von der Frequenz
\item im Frequenzbereich kann Systemantwort bestimmen
\end{itemize}
 
\[
Y(\omega) = H(\omega)\, X(\omega)
\]

\cite{SigSys:2022}




\end{frame}



\begin{frame}{Grenzfrequenz}

Bei der Grenzfrequenz fällt die Ausgangsspannung auf $1/\sqrt{2}$ des Eingangswerts.
Kann aus Übertragungsfunktion berechnet werden:

\begin{enumerate}
    \item Betrag bilden:
    \[
    |H(j\omega)| = \frac{1}{\sqrt{1 + (\omega RC)^2}}
    \]

    \item $-3\,\mathrm{dB}$-Bedingung anwenden:
    \[
    |H(j\omega_c)| = \frac{1}{\sqrt{2}}
    \]

    \item Gleichsetzen und lösen:
    \[
    \frac{1}{\sqrt{1 + (\omega_c RC)^2}} = \frac{1}{\sqrt{2}}
    \quad\Rightarrow\quad
    \omega_c = \frac{1}{RC}
    \]
\end{enumerate}

\cite{GrundlagenEtechnik:2006}

\end{frame}

\begin{frame}{Amplitudengang eines RC-Tiefpasses}

\begin{columns}[T]  % T = top aligned
    % Linke Spalte
    \begin{column}{0.30\textwidth}

\center
      Dämpfung von Ein-zu Ausgang 
            
\[
    |H(\omega)| = \frac{1}{\sqrt{1 + (\omega RC)^2}}
\]

 bei fg : Amplitude fällt auf -3 dB


    \end{column}

    % Rechte Spalte
    \begin{column}{0.55\textwidth}
       
            \begin{figure}
            \centering
            \includegraphics[width=\textwidth]{images/AmplitudenGang.png}
   
            \caption{Amp.gang RC-TP}
        \end{figure}
    
    \end{column}
\end{columns}
\cite{TietzeSchenk:2019}

\end{frame}


\begin{frame}{Phasengang eines RC-Tiefpasses}

\begin{columns}[T]  % T = top aligned
    % Linke Spalte
    \begin{column}{0.30\textwidth}

\center
      Phasenverschiebung von Ein-zu Ausgang
            
\[
            \varphi(\omega) = -\arctan(\omega RC)
        \]

 bei fg : Verschiebung von -45°

% Niederige Frequenzen kaum verschiebung
% hohe Frequenzen verschiebung nähert sich 90°
% Phase hinkt hinterher, Kapazität Spannung zu spät




    \end{column}

    % Rechte Spalte
    \begin{column}{0.55\textwidth}
       
            \begin{figure}
            \centering
            \includegraphics[width=\textwidth]{images/PhasengangRC.png}
   
            \caption{Ph.gang RC-TP}
        \end{figure}
    
    \end{column}
\end{columns}

\cite{TietzeSchenk:2019}

\end{frame}



\begin{frame}{Bodediagramm}
Darstellung beider Verhalten auf einen Blick

    \begin{figure}
        \centering
        \includegraphics[width=0.8\textwidth]{images/BodediagrammRC_markiert.png}
        \caption{Bodediagramm eines RC-Tiefpasses}
    \end{figure}
\end{frame}