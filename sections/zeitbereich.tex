


\begin{frame}{Tiefpass im Zeitbereich}

 Systembeschreibung mit DGL
% Noch Sichereren Dialog erstellen!!!!

\begin{minipage}{0.58\textwidth}
    Maschengleichung
    \[
        -u_{\mathrm{e}}(t) + u_{\mathrm{R}}(t) + u_{\mathrm{C}}(t) = 0
    \]

    % Stromgleichung

% Ohmsches Gesetz

% Kondensatorgesetz

%\subsection*{Einsetzen in die Maschengleichung}

%\subsection*{Differentialgleichung des RC-Tiefpasses}

    Differentialgleichung
    \[
        RC\, \frac{\mathrm{d}u_{\mathrm{C}}(t)}{\mathrm{d}t} + u_{\mathrm{C}}(t) = u_{\mathrm{e}}(t)
    \]

\cite{SigSys:2022}

    %Integration beider Seiten:


%Umgestellt nach der Ausgangsspannung:
\end{minipage}
\hfill
\begin{minipage}{0.38\textwidth}
    \centering
    \includegraphics[width=\textwidth]{images/RC-Schaltung.png}

    {\footnotesize RC‑Tiefpass-Schaltung \par}
    {\footnotesize Quelle: \cite{wikipediaRCGlied:2026} \par}
\end{minipage}

Berechnungen von DGL oft komplex und aufwendig

\end{frame}

\begin{frame}{Sprungantwort}

        \centering
        \includegraphics[width=\textwidth]{images/SprungantwortRC.png}
        {\footnotesize Sprungsantwort RC‑Tiefpass \par}
    

\cite{TietzeSchenk:2019}

\end{frame}

\begin{frame}{Impulsantwort}

    \centering
    \includegraphics[width=\textwidth]{images/ImpulsantwortRC.png}
    {\footnotesize Impulsantwort RC‑Tiefpass \par}

\cite{SigSys:2022}

\end{frame}


\begin{frame}{Darstellung von Signalen als Impulsfolge}
    
    \begin{figure}
        \centering
        \includegraphics[width=0.8\textwidth]{images/SinusDirac.png}
        \caption{Sinus als Dirac-Impulsfolge}
    \end{figure}
\cite{Faltung:2026}
    
\end{frame}

\begin{frame}{Übertragungsfunktion und Impulsantwort}

    Impulsantwort ist die Inverse Laplace-Transformation der Übertragungsfunktion:
\[
H(s) = \mathcal{L}\{h(t)\}
\]

\[
h(t) = \mathcal{L}^{-1}\{H(s)\}
\]

\cite{SigSys:2022}

\end{frame}


\begin{frame}{Faltung als Beschreibung im Zeitbereich}
    Berechnung Ausgangssignal mit Eingangssignal und Impulsantwort: 

\[
y(t) = (x * h)(t)
\]

\[
y(t) = \int_{-\infty}^{\infty} x(\tau)\, h(t - \tau)\, d\tau
\]




\cite{Faltung:2026}


\begin{figure}
        \centering
        \includegraphics[width=0.8\textwidth]{images/Faltung.png}
        
        {\footnotesize Graphische Faltung \par}
    {\footnotesize Quelle: \cite{GraphischeFaltung:2024} \par}
    \end{figure}

%Signal spiegeln und verschiebern
% beide signal multiplizieren und integrieren über alle Zeitpunkte


\end{frame}












