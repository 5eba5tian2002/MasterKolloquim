    \begin{frame}{Tiefpass im Zeitbereich}
        \begin{itemize}
        \item Beschreibung über Differentialgleichung
        \item Impulsantwort
        \item Faltung
        \item Sprungantwort
        
    \end{itemize}
\end{frame}


\begin{frame}{Systembeschreibung mit DGL}

\begin{minipage}{0.58\textwidth}
    Maschengleichung
    \[
        -u_{\mathrm{e}}(t) + u_{\mathrm{R}}(t) + u_{\mathrm{C}}(t) = 0
    \]

    % Stromgleichung

% Ohmsches Gesetz

% Kondensatorgesetz

%\subsection*{Einsetzen in die Maschengleichung}

%\subsection*{Differentialgleichung des RC-Tiefpasses}

    Differentialgleichung
    \[
        RC\, \frac{\mathrm{d}u_{\mathrm{C}}(t)}{\mathrm{d}t} + u_{\mathrm{C}}(t) = u_{\mathrm{e}}(t)
    \]

    %Integration beider Seiten:


%Umgestellt nach der Ausgangsspannung:
\end{minipage}
\hfill
\begin{minipage}{0.38\textwidth}
    \centering
    \includegraphics[width=\textwidth]{images/RC-Schaltung.png}

    {\footnotesize RC‑Tiefpass-Schaltung \par}
    {\footnotesize Quelle: \cite{wikipediaRCGlied:2026} \par}
\end{minipage}

Berechnungen von DGL oft komplex und aufwendig

\end{frame}

\begin{frame}{Impulsantwort}

    \centering
    \includegraphics[width=\textwidth]{images/ImpulsantwortRC.png}
    {\footnotesize RC‑Tiefpass-Schaltung \par}

\end{frame}


\begin{frame}{Darstellung von Signalen als Impulsfolge}
    
    \begin{figure}
        \centering
        \includegraphics[width=0.8\textwidth]{images/SinusDirac.png}
        \caption{Sinus als Dirac-Impulsfolge}
    \end{figure}
    
\end{frame}

\begin{frame}{Übertragungsfunktion und Impulsantwort}

    Impulsantwort ist die Inverse Laplace-Transformation der Übertragungsfunktion:
\[
H(s) = \mathcal{L}\{h(t)\}
\]

\[
h(t) = \mathcal{L}^{-1}\{H(s)\}
\]

\end{frame}


\begin{frame}{Faltung im Zeitkontinuierlichen Bereich}
    Berechnung Ausgangssignal mit Eingangssignal und Impulsantwort: 

\[
y(t) = (x * h)(t)
\]

\[
y(t) = \int_{-\infty}^{\infty} x(\tau)\, h(t - \tau)\, d\tau
\]

\end{frame}




\begin{frame}{Sprungantwort}

    \begin{figure}
        \centering
        \includegraphics[width=0.8\textwidth]{images/SprungantwortRC.png}
        \caption{Sprungsantwort eines RC-Tiefpasses}
    \end{figure}

\end{frame}







