\begin{frame}{Tiefpass im Frequenzbereich}
    \begin{itemize}
        \item Beschreibung über die Übertragungsfunktion:
        \[
            H(\omega) = \frac{1}{1 + j\omega RC}
        \]
        \item Grenzfrequenz:
        \[
            \omega_c = \frac{1}{RC}
        \]
        \item Amplitudengang:
        \[
            |H(\omega)| = \frac{1}{\sqrt{1 + (\omega RC)^2}}
        \]
        \item Phasengang:
        \[
            \varphi(\omega) = -\arctan(\omega RC)
        \]
        \item Tiefpass dämpft hohe Frequenzen und lässt niedrige passieren
    \end{itemize}
     \begin{itemize}
        \item Übertragungsfunktion
        \item Amplitudengang
        \item Phasengang
        \item Bodediagramm
    \end{itemize}
\end{frame}

\begin{frame}{Übertragungsfunktion eines Tiefpasses}

  \begin{itemize}
    \item Beschreibt das Verhältnis von Ausgangs- zu Eingangssignal im Frequenzbereich.
      \[
        H(s) = \frac{Y(s)}{X(s)}
      \]
  \end{itemize}

  U2 zu U1 bei ....


 


  

\end{frame}


\begin{frame}{Bodediagramm}

    \begin{figure}
        \centering
        \includegraphics[width=0.8\textwidth]{images/BodediagrammRC_markiert.png}
        \caption{Bodediagramm eines RC-Tiefpasses}
    \end{figure}
\end{frame}